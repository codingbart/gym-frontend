\documentclass[../../spr.tex]{subfiles}

\begin{document}

\section{Literatura}
    \begin{itemize}
        \item React – dokumentacja oficjalna, \url{https://react.dev/}, dostęp: 18.06.2025.
        \item Vite – dokumentacja oficjalna, \url{https://vite.dev/}, dostęp: 18.06.2025.
        \item Vitest – dokumentacja oficjalna, \url{https://vitest.dev/}, dostęp: 18.06.2025.
        \item React Router – dokumentacja oficjalna, \url{https://reactrouter.com/}, dostęp: 18.06.2025.
        \item Orval – dokumentacja oficjalna, \url{https://orval.dev}, dostęp: 18.06.2025.
        \item AntDesign – dokumentacja oficjalna, \url{https://ant.design/}, dostęp: 18.06.2025.
        \item Chart.js – dokumentacja oficjalna, \url{https://www.chartjs.org/}, dostęp: 18.06.2025.
        \item Keycloak JS – dokumentacja oficjalna, \url{https://www.keycloak.org/securing-apps/javascript-adapter}, dostęp: 18.06.2025.
        \item Tailwind CSS – dokumentacja oficjalna, \url{https://tailwindcss.com/}, dostęp: 18.06.2025.
        \item Day.js – dokumentacja oficjalna, \url{https://day.js.org/}, dostęp: 18.06.2025.
        \item Pawełkiwicz, Mateusz – wykłady z przedmiotu \textit{Zaawansowane aplikacje frontendowe}, Politechnika Świętokrzyska, semestr letni 2025:
        \begin{itemize}
            \item \textit{Wprowadzenie do Frontend Developmentu},
            \item \textit{Narzędzia i Frameworki},
            \item \textit{Stylowanie i układ strony},
            \item \textit{Zaawansowane techniki CSS},
            \item \textit{Podstawy JavaScript},
            \item \textit{Nowoczesny JavaScript, narzędzia i programowanie asynchroniczne},
            \item \textit{Frameworki i Biblioteki Frontendowe – React, Angular, Vue.js},
            \item \textit{Testowanie aplikacji frontendowych}.
        \end{itemize}
    \end{itemize}
\end{document}