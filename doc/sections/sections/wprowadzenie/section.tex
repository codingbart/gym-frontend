\documentclass[../../spr.tex]{subfiles}

\begin{document}

\section{Wprowadzenie}

\subsection{Krótki opis aplikacji}
Celem projektu było stworzenie aplikacji służącej do zarządzanie siłownią. System ten został zaprojektowany z myślą o czterech różnych typach użytkowników: kliencie, pracowniku, trenerze oraz menadżerze.

Aplikacja umożliwia m.in.:
\begin{itemize}
    \item śledzenie postępów treningowych przez użytkownika,
    \item przegląd i edycję zaplanowanych sesji treningowych,
    \item zapisywanie przebiegu własnych treningów,
    \item odnawianie karnetu,
    \item wgląd w listę klientów i pracowników (dla kadry zarządzającej),
    \item zarządzanie salami treningowymi,
    \item tworzenie i edytowanie kont pracowników oraz klientów.
\end{itemize}

\subsection{Wykorzystane technologie oraz narzędzia}
Do stworzenia aplikacji wykorzystano następujące technologie i narzędzia:
\begin{itemize}
    \item \textbf{TypeScript} – nadzbiór języka JavaScript umożliwiający statyczne typowanie,
    \item \textbf{Vite} – bundler i środowisko uruchomieniowe do 
    aplikacji frontendowych,
    \item \textbf{Vitest} – framework testowy, zintegrowany z ekosystemem Vite,
    \item \textbf{React} – biblioteka JavaScript służąca do budowy
    interfejsów użytkownika,
    \item \textbf{React Router} – biblioteka do zarządzania nawigacją i routingiem w aplikacjach React,
    \item \textbf{Axios} – biblioteka do obsługi zapytań HTTP,
    wykorzystywana do komunikacji z backendem przez REST API,
    \item \textbf{OpenAPI} – standard do opisywania interfejsów REST API,
    \item \textbf{Orval} – narzędzie generujące klienta API na podstawie specyfikacji OpenAPI,
    ułatwiające komunikację z backendem,
    \item \textbf{Figma} – narzędzie do projektowania interfejsów użytkownika i prototypowania,
    \item \textbf{Task} – narzędzie do automatyzacji zadań deweloperskich,
    \item \textbf{Git} – system kontroli wersji wspierający zarządzanie kodem
    źródłowym oraz pracę zespołową,
    \item \textbf{GitHub} – serwis internetowy umożliwiający przechowywanie i zarządzanie repozytoriami Git,
    \item \textbf{GitHub Actions} – platforma CI/CD umożliwiająca automatyzację
    procesów związanych z budowaniem, testowaniem i wdrażaniem aplikacji,
    \item \textbf{Renovate} – bot do automatycznego aktualizowania zależności projektu,
    \item \textbf{Nix} – narzędzie do zarządzania środowiskiem programistycznym w systemie NixOS,
    \item \textbf{AntDesign} – biblioteka komponentów UI dla React,
    \item \textbf{Chart.js} – biblioteka do tworzenia dynamicznych wykresów,
    \item \textbf{Keycloak JS} – biblioteka służąca do integracji aplikacji frontendowej z systemem \textit{Keycloak} do autoryzacji użytkowników,
    \item \textbf{Tailwind CSS} – biblioteka CSS do szybkiego stylowania interfejsów użytkownika,
    \item \textbf{Day.js} – biblioteka do obsługi dat i czasu.
\end{itemize}

\end{document}