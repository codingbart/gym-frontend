\documentclass[../../spr.tex]{subfiles}

\begin{document}

\section{Wprowadzenie/Cel laboratorium}

\subsection{Krótki opis aplikacji}
Celem projektu było stworzenie aplikacji wspomagającej zarządzanie siłownią. System ten został zaprojektowany z myślą o czterech różnych typach użytkowników: kliencie, pracowniku, trenerze oraz menadżerze. 

Aplikacja umożliwia m.in.:
\begin{itemize}
    \item śledzenie postępów treningowych przez użytkownika,
    \item przegląd i edycję zaplanowanych sesji treningowych,
    \item zapisywanie przebiegu własnych treningów,
    \item odnawianie karnetu,
    \item wgląd w listę klientów i pracowników (dla kadry zarządzającej),
    \item zarządzanie salami treningowymi,
    \item tworzenie i edytowanie kont pracowników oraz klientów.
\end{itemize}

\subsection{Wykorzystane technologie oraz narzędzia}
Do stworzenia aplikacji wykorzystano następujące technologie i narzędzia:
\begin{itemize}
    \item \textbf{Vite.js} – szybki bundler i środowisko uruchomieniowe do 
    aplikacji frontendowych,
    \item \textbf{React} – biblioteka JavaScript służąca do budowy
    interfejsów użytkownika w sposób komponentowy,
    \item \textbf{TypeScript} – nadzbiór JavaScriptu umożliwiający statyczne typowanie,
    co poprawia czytelność i niezawodność kodu,
    \item \textbf{Axios} – biblioteka do obsługi zapytań HTTP,
    wykorzystywana do komunikacji z backendem przez REST API,
    \item \textbf{Orval} – narzędzie generujące klienta API na podstawie specyfikacji OpenAPI,
    ułatwiające komunikację z backendem,
    \item \textbf{OpenAPI} – standard do opisywania REST API w sposób zrozumiały zarówno dla ludzi,
    jak i maszyn,
    \item \textbf{Task} – narzędzie do automatyzacji zadań deweloperskich,
    takich jak uruchamianie serwera, testy, czy budowanie aplikacji,
    \item \textbf{Git} – system kontroli wersji wspierający zarządzanie kodem
    źródłowym oraz pracę zespołową,
    \item \textbf{GitHub Actions} – platforma CI/CD umożliwiająca automatyzację
    procesów związanych z budowaniem, testowaniem i wdrażaniem aplikacji,
    \item \textbf{Renovate} – bot do automatycznego aktualizowania zależności projektu,\\
    co zwiększa bezpieczeństwo i aktualność środowiska,
    \item \textbf{Nix} - narzędzie do zarządzania środowiskiem programistycznym,
    które zapewnia spójność i powtarzalność konfiguracji,
    \item \textbf{AntDesign} – biblioteka komponentów UI,
    która przyspiesza proces tworzenia estetycznych i funkcjonalnych interfejsów użytkownika,
    \item \textbf{Node.js} – środowisko uruchomieniowe JavaScript,
    które umożliwia tworzenie aplikacji serwerowych i narzędzi deweloperskich.
\end{itemize}
\end{document}